% Options for packages loaded elsewhere
\PassOptionsToPackage{unicode}{hyperref}
\PassOptionsToPackage{hyphens}{url}
%
\documentclass[
]{article}
\title{Snapshot report on the 5Ws}
\usepackage{etoolbox}
\makeatletter
\providecommand{\subtitle}[1]{% add subtitle to \maketitle
  \apptocmd{\@title}{\par {\large #1 \par}}{}{}
}
\makeatother
\subtitle{First quarter 2022}
\author{Myanmar Food Security Cluster}
\date{2022-05-06}

\usepackage{amsmath,amssymb}
\usepackage{lmodern}
\usepackage{iftex}
\ifPDFTeX
  \usepackage[T1]{fontenc}
  \usepackage[utf8]{inputenc}
  \usepackage{textcomp} % provide euro and other symbols
\else % if luatex or xetex
  \usepackage{unicode-math}
  \defaultfontfeatures{Scale=MatchLowercase}
  \defaultfontfeatures[\rmfamily]{Ligatures=TeX,Scale=1}
\fi
% Use upquote if available, for straight quotes in verbatim environments
\IfFileExists{upquote.sty}{\usepackage{upquote}}{}
\IfFileExists{microtype.sty}{% use microtype if available
  \usepackage[]{microtype}
  \UseMicrotypeSet[protrusion]{basicmath} % disable protrusion for tt fonts
}{}
\makeatletter
\@ifundefined{KOMAClassName}{% if non-KOMA class
  \IfFileExists{parskip.sty}{%
    \usepackage{parskip}
  }{% else
    \setlength{\parindent}{0pt}
    \setlength{\parskip}{6pt plus 2pt minus 1pt}}
}{% if KOMA class
  \KOMAoptions{parskip=half}}
\makeatother
\usepackage{xcolor}
\IfFileExists{xurl.sty}{\usepackage{xurl}}{} % add URL line breaks if available
\IfFileExists{bookmark.sty}{\usepackage{bookmark}}{\usepackage{hyperref}}
\hypersetup{
  pdftitle={Snapshot report on the 5Ws},
  pdfauthor={Myanmar Food Security Cluster},
  hidelinks,
  pdfcreator={LaTeX via pandoc}}
\urlstyle{same} % disable monospaced font for URLs
\usepackage[margin=1in]{geometry}
\usepackage{longtable,booktabs,array}
\usepackage{calc} % for calculating minipage widths
% Correct order of tables after \paragraph or \subparagraph
\usepackage{etoolbox}
\makeatletter
\patchcmd\longtable{\par}{\if@noskipsec\mbox{}\fi\par}{}{}
\makeatother
% Allow footnotes in longtable head/foot
\IfFileExists{footnotehyper.sty}{\usepackage{footnotehyper}}{\usepackage{footnote}}
\makesavenoteenv{longtable}
\usepackage{graphicx}
\makeatletter
\def\maxwidth{\ifdim\Gin@nat@width>\linewidth\linewidth\else\Gin@nat@width\fi}
\def\maxheight{\ifdim\Gin@nat@height>\textheight\textheight\else\Gin@nat@height\fi}
\makeatother
% Scale images if necessary, so that they will not overflow the page
% margins by default, and it is still possible to overwrite the defaults
% using explicit options in \includegraphics[width, height, ...]{}
\setkeys{Gin}{width=\maxwidth,height=\maxheight,keepaspectratio}
% Set default figure placement to htbp
\makeatletter
\def\fps@figure{htbp}
\makeatother
\setlength{\emergencystretch}{3em} % prevent overfull lines
\providecommand{\tightlist}{%
  \setlength{\itemsep}{0pt}\setlength{\parskip}{0pt}}
\setcounter{secnumdepth}{-\maxdimen} % remove section numbering
\usepackage{booktabs}
\usepackage{longtable}
\usepackage{array}
\usepackage{multirow}
\usepackage{wrapfig}
\usepackage{float}
\usepackage{colortbl}
\usepackage{pdflscape}
\usepackage{tabu}
\usepackage{threeparttable}
\usepackage{threeparttablex}
\usepackage[normalem]{ulem}
\usepackage{makecell}
\usepackage{xcolor}
\ifLuaTeX
  \usepackage{selnolig}  % disable illegal ligatures
\fi

\begin{document}
\maketitle

\hypertarget{summary-of-achievements}{%
\subsection{Summary of achievements}\label{summary-of-achievements}}

Beneficiaries of humanitarian action formed 99.84\% of the 129,235
beneficiaries in the first quarter of 2022. The remainder were reached
through development interventions.

To recall, the Food Security Cluster's strategic objectives for 2022
are:

\begin{itemize}
\tightlist
\item
  SO1: IDPs have equitable access to sufficient, safe and nutritious
  food (either in-kind or through food assistance)
\item
  SO2: Vulnerable persons (excl. IDPs) have equitable access to
  sufficient, safe and nutritious food (either in-kind or through food
  assistance)
\item
  SO3: Restore, protect and improve livelihoods and resilience
\end{itemize}

\begin{longtable}[]{@{}
  >{\centering\arraybackslash}p{(\columnwidth - 6\tabcolsep) * \real{0.17}}
  >{\centering\arraybackslash}p{(\columnwidth - 6\tabcolsep) * \real{0.21}}
  >{\centering\arraybackslash}p{(\columnwidth - 6\tabcolsep) * \real{0.19}}
  >{\centering\arraybackslash}p{(\columnwidth - 6\tabcolsep) * \real{0.14}}@{}}
\caption{2022/Q1 beneficiaries by strategic objective}\tabularnewline
\toprule
\begin{minipage}[b]{\linewidth}\centering
strat\_obj
\end{minipage} & \begin{minipage}[b]{\linewidth}\centering
Humanitarian
\end{minipage} & \begin{minipage}[b]{\linewidth}\centering
Development
\end{minipage} & \begin{minipage}[b]{\linewidth}\centering
Total
\end{minipage} \\
\midrule
\endfirsthead
\toprule
\begin{minipage}[b]{\linewidth}\centering
strat\_obj
\end{minipage} & \begin{minipage}[b]{\linewidth}\centering
Humanitarian
\end{minipage} & \begin{minipage}[b]{\linewidth}\centering
Development
\end{minipage} & \begin{minipage}[b]{\linewidth}\centering
Total
\end{minipage} \\
\midrule
\endhead
SO\_1 & 98,106 & 0 & 98,106 \\
SO\_3 & 30,926 & 203 & 31,129 \\
Total & 129,032 & 203 & 129,235 \\
\bottomrule
\end{longtable}

In terms of activities, the number of beneficiaries reached is heavily
skewed towards food distributions. More than 53\% of beneficiaries in
2022/Q1 have been reached by this activity.

\begin{longtable}[]{@{}
  >{\centering\arraybackslash}p{(\columnwidth - 10\tabcolsep) * \real{0.39}}
  >{\centering\arraybackslash}p{(\columnwidth - 10\tabcolsep) * \real{0.13}}
  >{\centering\arraybackslash}p{(\columnwidth - 10\tabcolsep) * \real{0.13}}
  >{\centering\arraybackslash}p{(\columnwidth - 10\tabcolsep) * \real{0.13}}
  >{\centering\arraybackslash}p{(\columnwidth - 10\tabcolsep) * \real{0.11}}
  >{\centering\arraybackslash}p{(\columnwidth - 10\tabcolsep) * \real{0.11}}@{}}
\caption{Breakdown of beneficiaries by activity in
2022/Q1}\tabularnewline
\toprule
\begin{minipage}[b]{\linewidth}\centering
activity
\end{minipage} & \begin{minipage}[b]{\linewidth}\centering
ben\_SO\_1
\end{minipage} & \begin{minipage}[b]{\linewidth}\centering
ben\_SO\_3
\end{minipage} & \begin{minipage}[b]{\linewidth}\centering
ben\_SO\_2
\end{minipage} & \begin{minipage}[b]{\linewidth}\centering
Total
\end{minipage} & \begin{minipage}[b]{\linewidth}\centering
\%\_ben
\end{minipage} \\
\midrule
\endfirsthead
\toprule
\begin{minipage}[b]{\linewidth}\centering
activity
\end{minipage} & \begin{minipage}[b]{\linewidth}\centering
ben\_SO\_1
\end{minipage} & \begin{minipage}[b]{\linewidth}\centering
ben\_SO\_3
\end{minipage} & \begin{minipage}[b]{\linewidth}\centering
ben\_SO\_2
\end{minipage} & \begin{minipage}[b]{\linewidth}\centering
Total
\end{minipage} & \begin{minipage}[b]{\linewidth}\centering
\%\_ben
\end{minipage} \\
\midrule
\endhead
food distribution & 69,206 & 0 & 0 & 69,206 & 53.55 \\
crop, vegetable and seed kits & 0 & 29,026 & 0 & 29,026 & 22.46 \\
multi-purpose cash transfer & 28,900 & 0 & 0 & 28,900 & 22.36 \\
IGA and small grants & 0 & 1,900 & 0 & 1,900 & 1.47 \\
vocational training & 0 & 203 & 0 & 203 & 0.16 \\
\bottomrule
\end{longtable}

31\% of beneficiaries were reached by activities where nutrition was
mainstreamed. This is highly encouraging. As the year progresses, it
will be important to collect more details about how exactly nutrition
has been mainstreamed so that coordination with the Nutrition Cluster
may be improved.

\begin{longtable}[]{@{}
  >{\centering\arraybackslash}p{(\columnwidth - 8\tabcolsep) * \real{0.41}}
  >{\centering\arraybackslash}p{(\columnwidth - 8\tabcolsep) * \real{0.09}}
  >{\centering\arraybackslash}p{(\columnwidth - 8\tabcolsep) * \real{0.09}}
  >{\centering\arraybackslash}p{(\columnwidth - 8\tabcolsep) * \real{0.22}}
  >{\centering\arraybackslash}p{(\columnwidth - 8\tabcolsep) * \real{0.18}}@{}}
\caption{Breakdown of beneficiaries by status of nutrition
mainstreaming}\tabularnewline
\toprule
\begin{minipage}[b]{\linewidth}\centering
was\_nutrition\_mainstreamed\_in\_activity
\end{minipage} & \begin{minipage}[b]{\linewidth}\centering
SO\_1
\end{minipage} & \begin{minipage}[b]{\linewidth}\centering
SO\_3
\end{minipage} & \begin{minipage}[b]{\linewidth}\centering
total\_beneficiaries
\end{minipage} & \begin{minipage}[b]{\linewidth}\centering
\%\_beneficiaries
\end{minipage} \\
\midrule
\endfirsthead
\toprule
\begin{minipage}[b]{\linewidth}\centering
was\_nutrition\_mainstreamed\_in\_activity
\end{minipage} & \begin{minipage}[b]{\linewidth}\centering
SO\_1
\end{minipage} & \begin{minipage}[b]{\linewidth}\centering
SO\_3
\end{minipage} & \begin{minipage}[b]{\linewidth}\centering
total\_beneficiaries
\end{minipage} & \begin{minipage}[b]{\linewidth}\centering
\%\_beneficiaries
\end{minipage} \\
\midrule
\endhead
Yes & 10,516 & 29,229 & 39,745 & 30.75 \\
No & 87,590 & 1,900 & 89,490 & 69.25 \\
\bottomrule
\end{longtable}

\hypertarget{implementation-of-information-sharing-protocols}{%
\subsubsection{Implementation of Information Sharing
Protocols}\label{implementation-of-information-sharing-protocols}}

The newly approved ICCG Information Sharing Protocols have been
implemented in this report to support the safe, ethical and effective
management of data within Myanmar. This implementation is most evident
in the use of partner pseudonymisation in this report. Partner names
have been replaced by tokens in this report. These tokens are stored in
a secure translation table on OCHA servers outside of Myanmar, where
they may not be requisitioned by authorities. For more information,
please read the full text of the protocols in either
\href{https://www.dropbox.com/s/11kv6cvnbvx9hbe/information_sharing_protocol_220323.pdf?dl=0}{English}
or
\href{https://www.dropbox.com/s/lqwvo7k80s1xnjj/SE\%20partner\%20data\%20protection_final_220323\%20translation.pdf?dl=0}{Myanmar}.

\hypertarget{geographies}{%
\subsection{1. Geographies}\label{geographies}}

\hypertarget{statewise-breakdowns}{%
\subsubsection{1.1 Statewise breakdowns}\label{statewise-breakdowns}}

As in 2021, the number of beneficiaries reached has been heavily biased
towards relatively few areas, which is not appropriate for a unionwide
response. A total of 19 townships have been reached across 4
states/regions.

\includegraphics{fsc_q1_2022_southeast_word_files/figure-latex/barplot-state-region-1.pdf}

The table below outlines the number of beneficiaries reached by
state/region in both 2021 and 2022/Q1. In 2021, the number of
beneficiaries reached was lowest in Tanintharyi region, similarly more
pronounced in 2022/Q1.

\begin{longtable}[]{@{}
  >{\centering\arraybackslash}p{(\columnwidth - 12\tabcolsep) * \real{0.13}}
  >{\centering\arraybackslash}p{(\columnwidth - 12\tabcolsep) * \real{0.20}}
  >{\centering\arraybackslash}p{(\columnwidth - 12\tabcolsep) * \real{0.12}}
  >{\centering\arraybackslash}p{(\columnwidth - 12\tabcolsep) * \real{0.20}}
  >{\centering\arraybackslash}p{(\columnwidth - 12\tabcolsep) * \real{0.12}}
  >{\centering\arraybackslash}p{(\columnwidth - 12\tabcolsep) * \real{0.08}}
  >{\centering\arraybackslash}p{(\columnwidth - 12\tabcolsep) * \real{0.15}}@{}}
\caption{Skew in Q1 2022 geographic reach, comparison with 2021
data}\tabularnewline
\toprule
\begin{minipage}[b]{\linewidth}\centering
State
\end{minipage} & \begin{minipage}[b]{\linewidth}\centering
Beneficiaries\_2021
\end{minipage} & \begin{minipage}[b]{\linewidth}\centering
\%\_ben\_2021
\end{minipage} & \begin{minipage}[b]{\linewidth}\centering
Beneficiaries\_2022
\end{minipage} & \begin{minipage}[b]{\linewidth}\centering
\%\_ben\_2022
\end{minipage} & \begin{minipage}[b]{\linewidth}\centering
Target
\end{minipage} & \begin{minipage}[b]{\linewidth}\centering
\%\_target\_2022
\end{minipage} \\
\midrule
\endfirsthead
\toprule
\begin{minipage}[b]{\linewidth}\centering
State
\end{minipage} & \begin{minipage}[b]{\linewidth}\centering
Beneficiaries\_2021
\end{minipage} & \begin{minipage}[b]{\linewidth}\centering
\%\_ben\_2021
\end{minipage} & \begin{minipage}[b]{\linewidth}\centering
Beneficiaries\_2022
\end{minipage} & \begin{minipage}[b]{\linewidth}\centering
\%\_ben\_2022
\end{minipage} & \begin{minipage}[b]{\linewidth}\centering
Target
\end{minipage} & \begin{minipage}[b]{\linewidth}\centering
\%\_target\_2022
\end{minipage} \\
\midrule
\endhead
Kayah & 17,746 & 12.81 & 48,951 & 37.88 & 64,400 & 76.01 \\
Kayin & 68,108 & 49.17 & 48,698 & 37.68 & 96,320 & 50.56 \\
Mon & 48,181 & 34.78 & 31,191 & 24.14 & 35,000 & 89.12 \\
Tanintharyi & 4,476 & 3.23 & 395 & 0.31 & 54,410 & 0.73 \\
\bottomrule
\end{longtable}

\hypertarget{township-level-breakdowns}{%
\subsubsection{1.2 Township-level
breakdowns}\label{township-level-breakdowns}}

Just 4 townships (listed in the table below) contained 63\% of all
beneficiaries. Reached 4 townships in Mon State which were not targeted
in HRP 2022. Only few number of beneficiaries were reached in Loikaw and
Ye townships which had a large number of target.

\begin{longtable}[]{@{}
  >{\centering\arraybackslash}p{(\columnwidth - 10\tabcolsep) * \real{0.17}}
  >{\centering\arraybackslash}p{(\columnwidth - 10\tabcolsep) * \real{0.18}}
  >{\centering\arraybackslash}p{(\columnwidth - 10\tabcolsep) * \real{0.11}}
  >{\centering\arraybackslash}p{(\columnwidth - 10\tabcolsep) * \real{0.19}}
  >{\centering\arraybackslash}p{(\columnwidth - 10\tabcolsep) * \real{0.21}}
  >{\centering\arraybackslash}p{(\columnwidth - 10\tabcolsep) * \real{0.14}}@{}}
\caption{Townships reached by number of beneficiaries}\tabularnewline
\toprule
\begin{minipage}[b]{\linewidth}\centering
state
\end{minipage} & \begin{minipage}[b]{\linewidth}\centering
township
\end{minipage} & \begin{minipage}[b]{\linewidth}\centering
target
\end{minipage} & \begin{minipage}[b]{\linewidth}\centering
beneficiaries
\end{minipage} & \begin{minipage}[b]{\linewidth}\centering
\%\_beneficiaries
\end{minipage} & \begin{minipage}[b]{\linewidth}\centering
\%\_reached
\end{minipage} \\
\midrule
\endfirsthead
\toprule
\begin{minipage}[b]{\linewidth}\centering
state
\end{minipage} & \begin{minipage}[b]{\linewidth}\centering
township
\end{minipage} & \begin{minipage}[b]{\linewidth}\centering
target
\end{minipage} & \begin{minipage}[b]{\linewidth}\centering
beneficiaries
\end{minipage} & \begin{minipage}[b]{\linewidth}\centering
\%\_beneficiaries
\end{minipage} & \begin{minipage}[b]{\linewidth}\centering
\%\_reached
\end{minipage} \\
\midrule
\endhead
Kayah & Hpruso & 4,000 & 30,215 & 23.38 & 755.4 \\
Kayin & Myawaddy & 5,000 & 20,843 & 16.13 & 416.9 \\
Kayah & Demoso & 25,000 & 16,315 & 12.62 & 65.26 \\
Kayin & Hpapun & 44,320 & 14,444 & 11.18 & 32.59 \\
Mon & Kyaikto & 10,000 & 8,284 & 6.41 & 82.84 \\
Kayin & Hlaingbwe & 15,000 & 7,778 & 6.02 & 51.85 \\
Mon & Chaungzon & 0 & 7,181 & 5.56 & 100 \\
Mon & Bilin & 5,000 & 6,031 & 4.67 & 120.6 \\
Mon & Paung & 0 & 4,254 & 3.29 & 100 \\
Kayin & Thandaunggyi & 7,000 & 2,811 & 2.18 & 40.16 \\
Mon & Mawlamyine & 0 & 2,691 & 2.08 & 100 \\
Mon & Kyaikmaraw & 0 & 2,570 & 1.99 & 100 \\
Kayah & Shadaw & 500 & 1,606 & 1.24 & 321.2 \\
Kayin & Kawkareik & 5,000 & 1,432 & 1.11 & 28.64 \\
Kayin & Kyainseikgyi & 5,000 & 1,390 & 1.08 & 27.8 \\
Kayah & Loikaw & 25,000 & 815 & 0.63 & 3.26 \\
Tanintharyi & Dawei & 5,667 & 314 & 0.24 & 5.54 \\
Mon & Ye & 10,000 & 180 & 0.14 & 1.8 \\
Tanintharyi & Yebyu & 4,751 & 81 & 0.06 & 1.7 \\
\bottomrule
\end{longtable}

\hypertarget{locations}{%
\subsubsection{1.3 Locations}\label{locations}}

A location refers to either an village, ward, IDP site or industrial
zone.

This first plot below is a histogram of location, by number of
beneficiaries. The vast majority of locations have only one activity
occurring within them. This is something to be monitored over the course
of the year, as it is assumed that a range of activities are required to
comprehensively meet the food security and livelihoods needs of targeted
communities. As it currently stands, the response is very broad, with
little depth.

\includegraphics{fsc_q1_2022_southeast_word_files/figure-latex/facet-locations-activities-1.pdf}

This second plot of locations is faceted by the number of partners --
this helps us check for potential overlaps. There are quite a large
number of locations with 2 partners in them. Locations with multiple
partners were from Rakhine, Kachin and Shan (North).

\includegraphics{fsc_q1_2022_southeast_word_files/figure-latex/histogram-locations-by-partner-1.pdf}

The greatest number of beneficiaries came from rural villages and
camp/IDP Sites. This runs counter to vulnerability patterns identified
by in both the IFPRI Household Welfare Survey and the FAO-WFP Food
Security Survey. Both surveys found that rural households were less food
secure and less resilient than urban ones.

\begin{longtable}[]{@{}
  >{\centering\arraybackslash}p{(\columnwidth - 10\tabcolsep) * \real{0.20}}
  >{\centering\arraybackslash}p{(\columnwidth - 10\tabcolsep) * \real{0.17}}
  >{\centering\arraybackslash}p{(\columnwidth - 10\tabcolsep) * \real{0.13}}
  >{\centering\arraybackslash}p{(\columnwidth - 10\tabcolsep) * \real{0.17}}
  >{\centering\arraybackslash}p{(\columnwidth - 10\tabcolsep) * \real{0.12}}
  >{\centering\arraybackslash}p{(\columnwidth - 10\tabcolsep) * \real{0.20}}@{}}
\caption{Breakdown of locations and beneficiaries}\tabularnewline
\toprule
\begin{minipage}[b]{\linewidth}\centering
rural\_or\_urban
\end{minipage} & \begin{minipage}[b]{\linewidth}\centering
location\_type
\end{minipage} & \begin{minipage}[b]{\linewidth}\centering
locations
\end{minipage} & \begin{minipage}[b]{\linewidth}\centering
beneficiaries
\end{minipage} & \begin{minipage}[b]{\linewidth}\centering
\%\_of\_ben
\end{minipage} & \begin{minipage}[b]{\linewidth}\centering
ben\_per\_location
\end{minipage} \\
\midrule
\endfirsthead
\toprule
\begin{minipage}[b]{\linewidth}\centering
rural\_or\_urban
\end{minipage} & \begin{minipage}[b]{\linewidth}\centering
location\_type
\end{minipage} & \begin{minipage}[b]{\linewidth}\centering
locations
\end{minipage} & \begin{minipage}[b]{\linewidth}\centering
beneficiaries
\end{minipage} & \begin{minipage}[b]{\linewidth}\centering
\%\_of\_ben
\end{minipage} & \begin{minipage}[b]{\linewidth}\centering
ben\_per\_location
\end{minipage} \\
\midrule
\endhead
Rural & Camp/IDP site & 52 & 28,603 & 22.24 & 550 \\
Rural & Village & 155 & 99,825 & 77.62 & 644 \\
Urban/Peri-Urban & Camp/IDP site & 1 & 104 & 0.08 & 104 \\
Urban/Peri-Urban & Village & 1 & 81 & 0.06 & 81 \\
\bottomrule
\end{longtable}

\hypertarget{activities}{%
\subsection{2. Activities}\label{activities}}

\hypertarget{progress-by-activity}{%
\subsubsection{2.1 Progress by activity}\label{progress-by-activity}}

The first grey line below shows the the approval of the IERP in June
2021 and the second red line shows the start of 2022.

\includegraphics{fsc_q1_2022_southeast_word_files/figure-latex/progress-facet-lineplot-1.pdf}

FFS and farmer training, and food cash for work/assets activities have
not been implemented in 2022/Q1. Income-generating activities, which
were very few in 2021, started to increase in 2022/Q1. Food
distributions (in-kind and CBT/CVA) continued to be the largest activity
from 2021 into 2022/Q1. Multi-purpose cash transfers was new activity
that was not present in 2021.

\hypertarget{agricultural-and-livelihoods-activities}{%
\subsubsection{2.2 Agricultural and livelihoods
activities}\label{agricultural-and-livelihoods-activities}}

24\% of all beneficiary frequencies pertained to agricultural
activities. As mentioned earlier, the vast majority of beneficiaries in
Q1 2022 were related to food distributions.

\begin{longtable}[]{@{}
  >{\centering\arraybackslash}p{(\columnwidth - 10\tabcolsep) * \real{0.27}}
  >{\centering\arraybackslash}p{(\columnwidth - 10\tabcolsep) * \real{0.18}}
  >{\centering\arraybackslash}p{(\columnwidth - 10\tabcolsep) * \real{0.20}}
  >{\centering\arraybackslash}p{(\columnwidth - 10\tabcolsep) * \real{0.09}}
  >{\centering\arraybackslash}p{(\columnwidth - 10\tabcolsep) * \real{0.13}}
  >{\centering\arraybackslash}p{(\columnwidth - 10\tabcolsep) * \real{0.13}}@{}}
\caption{Beneficiary frequencies reached by agricultural and
non-agricultrural activities}\tabularnewline
\toprule
\begin{minipage}[b]{\linewidth}\centering
agricultural\_activity
\end{minipage} & \begin{minipage}[b]{\linewidth}\centering
beneficiaries
\end{minipage} & \begin{minipage}[b]{\linewidth}\centering
\%\_beneficiaries
\end{minipage} & \begin{minipage}[b]{\linewidth}\centering
state
\end{minipage} & \begin{minipage}[b]{\linewidth}\centering
townships
\end{minipage} & \begin{minipage}[b]{\linewidth}\centering
partners
\end{minipage} \\
\midrule
\endfirsthead
\toprule
\begin{minipage}[b]{\linewidth}\centering
agricultural\_activity
\end{minipage} & \begin{minipage}[b]{\linewidth}\centering
beneficiaries
\end{minipage} & \begin{minipage}[b]{\linewidth}\centering
\%\_beneficiaries
\end{minipage} & \begin{minipage}[b]{\linewidth}\centering
state
\end{minipage} & \begin{minipage}[b]{\linewidth}\centering
townships
\end{minipage} & \begin{minipage}[b]{\linewidth}\centering
partners
\end{minipage} \\
\midrule
\endhead
no & 98,309 & 76.07 & 4 & 13 & 3 \\
yes & 30,926 & 23.93 & 3 & 9 & 2 \\
\bottomrule
\end{longtable}

Crop, vegetable and seed kits formed the largest group of agricultural
activities and reached 5,729 households (for a full breakdown by
agricultural activity, please refer to the plot below).

It will be important to review the results from the second quarter in
order to see if this pattern changes and agricultural household receive
sufficient assistance prior to the main rice planting season which
begins in May 2022. Still, the results are not encouraging and
agricultural activities have had a very limited reach.

\includegraphics{fsc_q1_2022_southeast_word_files/figure-latex/barplot-ag-activities-1.pdf}

\hypertarget{delivery-modalities}{%
\subsubsection{2.3 Delivery modalities}\label{delivery-modalities}}

The plots below, faceted by delivery modality, show the breakdown of
activities by delivery modality. All activities corresponded to only one
type of delivery modality.

\begin{longtable}[]{@{}
  >{\centering\arraybackslash}p{(\columnwidth - 8\tabcolsep) * \real{0.37}}
  >{\centering\arraybackslash}p{(\columnwidth - 8\tabcolsep) * \real{0.11}}
  >{\centering\arraybackslash}p{(\columnwidth - 8\tabcolsep) * \real{0.11}}
  >{\centering\arraybackslash}p{(\columnwidth - 8\tabcolsep) * \real{0.22}}
  >{\centering\arraybackslash}p{(\columnwidth - 8\tabcolsep) * \real{0.18}}@{}}
\caption{Percentage of beneficiaries reached by activity and delivery
modality}\tabularnewline
\toprule
\begin{minipage}[b]{\linewidth}\centering
Activity
\end{minipage} & \begin{minipage}[b]{\linewidth}\centering
In-kind
\end{minipage} & \begin{minipage}[b]{\linewidth}\centering
CBT/CVA
\end{minipage} & \begin{minipage}[b]{\linewidth}\centering
Service delivery
\end{minipage} & \begin{minipage}[b]{\linewidth}\centering
Beneficiaries
\end{minipage} \\
\midrule
\endfirsthead
\toprule
\begin{minipage}[b]{\linewidth}\centering
Activity
\end{minipage} & \begin{minipage}[b]{\linewidth}\centering
In-kind
\end{minipage} & \begin{minipage}[b]{\linewidth}\centering
CBT/CVA
\end{minipage} & \begin{minipage}[b]{\linewidth}\centering
Service delivery
\end{minipage} & \begin{minipage}[b]{\linewidth}\centering
Beneficiaries
\end{minipage} \\
\midrule
\endhead
food distribution & 100.0\% & NA & NA & 69,206 \\
crop, vegetable and seed kits & 100.0\% & NA & NA & 29,026 \\
multi-purpose cash transfer & NA & 100.0\% & NA & 28,900 \\
IGA and small grants & NA & 100.0\% & NA & 1,638 \\
vocational training & NA & NA & 100.0\% & 203 \\
\bottomrule
\end{longtable}

Both villages and IDP/Camp sites were predominated by in-kind
distributions whilst only villages were targeted with cash-based
interventions.

\includegraphics{fsc_q1_2022_southeast_word_files/figure-latex/facet-location-1.pdf}

This would perhaps imply that partners believe that markets were more
accessible from villages than camp/IDP sites. Other alternative
assumptions include donor preferences and logistical challenges in
bringing in-kind goods to villages. This remains a question to be
explored by the broader Food Security Cluster.

Below is a breakdown of percentage of beneficiaries reached by the
different delivery modalities, by state.

\includegraphics{fsc_q1_2022_southeast_word_files/figure-latex/delivery-modalities-stacked-bar-1.pdf}

\hypertarget{cash-based-programming}{%
\subsection{3. Cash-based programming}\label{cash-based-programming}}

\hypertarget{cash-transfer-values-per-household}{%
\subsubsection{3.1 Cash transfer values per
household}\label{cash-transfer-values-per-household}}

\includegraphics{fsc_q1_2022_southeast_word_files/figure-latex/usd-hhd-bin-barplot-1.pdf}

46\% of households received less than USD 40/month per transfer.
However, the most common transfer values were between USD 40/month and
USD 70/month, with 47\% of households receiving transfers in this range.
This aligns fairly well with 50\% of the Minimum Expenditure Basket for
food expenditures (USD 52.28/household/month). It should be noted,
however that the value of the Minimum Expenditure Basket (calculated for
2021) needs to be revised as the Food Security Cluster anticipates 40\%
inflation in 2022.

The table below shows the average USD values per transfer per household
by and total transfer values per activity in the first quarter of 2022.

\begin{longtable}[]{@{}
  >{\centering\arraybackslash}p{(\columnwidth - 6\tabcolsep) * \real{0.34}}
  >{\centering\arraybackslash}p{(\columnwidth - 6\tabcolsep) * \real{0.21}}
  >{\centering\arraybackslash}p{(\columnwidth - 6\tabcolsep) * \real{0.21}}
  >{\centering\arraybackslash}p{(\columnwidth - 6\tabcolsep) * \real{0.24}}@{}}
\caption{Only households which were reached by cash, hybrid or voucher
modalities are included}\tabularnewline
\toprule
\begin{minipage}[b]{\linewidth}\centering
activity
\end{minipage} & \begin{minipage}[b]{\linewidth}\centering
hhd\_frequencies
\end{minipage} & \begin{minipage}[b]{\linewidth}\centering
total\_value\_usd
\end{minipage} & \begin{minipage}[b]{\linewidth}\centering
avg\_transfer\_value
\end{minipage} \\
\midrule
\endfirsthead
\toprule
\begin{minipage}[b]{\linewidth}\centering
activity
\end{minipage} & \begin{minipage}[b]{\linewidth}\centering
hhd\_frequencies
\end{minipage} & \begin{minipage}[b]{\linewidth}\centering
total\_value\_usd
\end{minipage} & \begin{minipage}[b]{\linewidth}\centering
avg\_transfer\_value
\end{minipage} \\
\midrule
\endhead
multi-purpose cash transfer & 5,110 & 454,046 & 88.85 \\
IGA and small grants & 303 & 20,473 & 67.57 \\
\bottomrule
\end{longtable}

\hypertarget{cash-transfer-values-by-implementing-partner}{%
\subsubsection{3.2 Cash transfer values by implementing
partner}\label{cash-transfer-values-by-implementing-partner}}

The plots below show average cash transfer values by activity of the
partners who reached the most beneficiaries.

The x-axis shows the average value per person or per household,
depending on the activity and the colour indicates the number of
beneficiaries reached.

\includegraphics{fsc_q1_2022_southeast_word_files/figure-latex/partner-cash-values-1.pdf}

\hypertarget{cash-transfer-values-per-person}{%
\subsubsection{3.3 Cash transfer values per
person}\label{cash-transfer-values-per-person}}

The boxplots above shows the range of cash transfer values (all values
are per person, to facilitate comparability) by activity. The average
for reach activity is marked by the thick line in the middle of each
box. The leftmost and rightmost side of each box indicate the 25th and
75th percentile of transfer values, respectively. The length of each box
is a gauge for how much variation there is in the transfer values of
each activity.

\includegraphics{fsc_q1_2022_southeast_word_files/figure-latex/boxplot-activity-usd-per-person-1.pdf}

Additionally, each of the bubbles indicate an individual distribution,
with their position along the x-axis showing the USD per person value of
the distribution and the size of each bubble indicates the number of
beneficiaries reached.

Despite being the activity which reached the most beneficiaries, food
distributions have one of the tightest ranges of transfer values,
though, as will be explored further in the plot below and in the next
section, there are substantial outliers.

In the interactive scatterplot below, the x-axis indicates the number of
beneficiaries reached and the y-axis indicates the per person value of
each transfer. Each point is a distribution and the size of each point
indicates the number of beneficiaries reached. More details about each
distribution can be seen by hovering your cursor over each point.

\includegraphics{fsc_q1_2022_southeast_word_files/figure-latex/plotly-transfer-value-scatter-1.pdf}

Food for work/cash for assets and multi-purpose cash transfers had the
largest dispersions in the values of their transfers. For food for
work/cash for assets, there is one cluster largely below USD 5/person in
Sagaing and another of between USD 30 and USD 60 per person in Kachin
and Shan North.

As mentioned, food distributions had the tightest range of transfer
values, with the vast majority of distributions falling just below USD
10/person. However, it has outlying values that reached very large
groups of beneficiaries. This will be explored in the next section.

\hypertarget{a-closer-look-at-food-distributions}{%
\subsubsection{3.4 A closer look at food
distributions}\label{a-closer-look-at-food-distributions}}

The interactive plot below breaks down the range of USD per person cash
transfer values by state. Similar to the plot above, each point is a
distribution and more details about each distribution can be seen by
hovering your mouse over each point.

The red line indicates 50\% of the monthly expenditure basket (MEB) for
food (divided by 5 to get the figure per person). The vast majority of
transfers fall below this value.

\includegraphics{fsc_q1_2022_southeast_word_files/figure-latex/plotly-food-dist-range-1.pdf}

Kachin and Shan notably have several extreme outliers much higher than
the average for that state. Kayin, however, has a very large number of
beneficiaries who received less the USD 1/person. Distributions in Chin
had very consistent values as they were all implemented by the same
implementing partner.

The table below compares the different bins for cash transfer values of
food distributions with the minimum expenditure basket for food
established by the Cash Working Group. They have established a floor of
MMK 190,555 (or USD 114.55).

Overall, 1.44\% of food distribution beneficiaries have received at
least 100\% of the MEB and 10.05\% have received at least 50\% of the
MEB.

\begin{longtable}[]{@{}
  >{\centering\arraybackslash}p{(\columnwidth - 8\tabcolsep) * \real{0.22}}
  >{\centering\arraybackslash}p{(\columnwidth - 8\tabcolsep) * \real{0.21}}
  >{\centering\arraybackslash}p{(\columnwidth - 8\tabcolsep) * \real{0.21}}
  >{\centering\arraybackslash}p{(\columnwidth - 8\tabcolsep) * \real{0.21}}
  >{\centering\arraybackslash}p{(\columnwidth - 8\tabcolsep) * \real{0.16}}@{}}
\caption{Monthly cash-based transfer values by percentage of MEB
received }\tabularnewline
\toprule
\begin{minipage}[b]{\linewidth}\centering
usd\_person\_bin
\end{minipage} & \begin{minipage}[b]{\linewidth}\centering
avg\_pc\_of\_meb
\end{minipage} & \begin{minipage}[b]{\linewidth}\centering
avg\_usd\_month
\end{minipage} & \begin{minipage}[b]{\linewidth}\centering
beneficiaries
\end{minipage} & \begin{minipage}[b]{\linewidth}\centering
pc\_of\_ben
\end{minipage} \\
\midrule
\endfirsthead
\toprule
\begin{minipage}[b]{\linewidth}\centering
usd\_person\_bin
\end{minipage} & \begin{minipage}[b]{\linewidth}\centering
avg\_pc\_of\_meb
\end{minipage} & \begin{minipage}[b]{\linewidth}\centering
avg\_usd\_month
\end{minipage} & \begin{minipage}[b]{\linewidth}\centering
beneficiaries
\end{minipage} & \begin{minipage}[b]{\linewidth}\centering
pc\_of\_ben
\end{minipage} \\
\midrule
\endhead
\textless\$2 & 1.65 & 0.38 & 20,843 & 37.36 \\
\textgreater=\$2\_\textless\$4 & 10.56 & 2.42 & 25,000 & 44.81 \\
\textgreater=\$16\_\textless\$18 & 76.28 & 17.48 & 6,665 & 11.95 \\
\textgreater=\$20 & 139.4 & 31.95 & 3,284 & 5.89 \\
\bottomrule
\end{longtable}

However, a very large proportion of the beneficiaries reached were
between USD 8 and 10 per person, fairly close to 50\% of the MEB. The
50\% threshold is of interest because humanitarian assistance does not
aim to cover the full MEB and is intended to meet acute needs.

\includegraphics{fsc_q1_2022_southeast_word_files/figure-latex/barplot-2021-2022-1.pdf}

With reference to the plot above, the per person USD values in 2022 are
more consistent than in 2021, with more than 50\% of beneficiary
frequencies receiving between USD 8 and 10 per transfer. The average
transfer value for food distributions in 2021 was USD 7.36; in 2022/Q1,
it was USD 8.59.

\hypertarget{beneficiaries}{%
\subsection{4. Beneficiaries}\label{beneficiaries}}

\hypertarget{beneficiary-types}{%
\subsubsection{4.1 Beneficiary types}\label{beneficiary-types}}

74.71\% of beneficiaries were from the host or local community. 14.58\%
beneficiaries were IDPs.

\includegraphics{fsc_q1_2022_southeast_word_files/figure-latex/unnamed-chunk-1-1.pdf}

\hypertarget{evidence-of-food-insecurity-status}{%
\subsubsection{4.2 Evidence of food insecurity
status}\label{evidence-of-food-insecurity-status}}

Very few of the beneficiaries reached had evidence of their food
insecurity status. This makes it difficult to determine whether or not
food security interventions are truly reaching those most in need.

\begin{longtable}[]{@{}
  >{\centering\arraybackslash}p{(\columnwidth - 4\tabcolsep) * \real{0.38}}
  >{\centering\arraybackslash}p{(\columnwidth - 4\tabcolsep) * \real{0.22}}
  >{\centering\arraybackslash}p{(\columnwidth - 4\tabcolsep) * \real{0.24}}@{}}
\caption{Food insecurity status and evidence provided in
2022/Q1}\tabularnewline
\toprule
\begin{minipage}[b]{\linewidth}\centering
food\_insecurity\_status
\end{minipage} & \begin{minipage}[b]{\linewidth}\centering
beneficiaries
\end{minipage} & \begin{minipage}[b]{\linewidth}\centering
\%\_benficiaries
\end{minipage} \\
\midrule
\endfirsthead
\toprule
\begin{minipage}[b]{\linewidth}\centering
food\_insecurity\_status
\end{minipage} & \begin{minipage}[b]{\linewidth}\centering
beneficiaries
\end{minipage} & \begin{minipage}[b]{\linewidth}\centering
\%\_benficiaries
\end{minipage} \\
\midrule
\endhead
Moderately food insecure & 19,705 & 15.25 \\
Severely food insecure & 87,590 & 67.78 \\
No status provided & 21,940 & 16.98 \\
\bottomrule
\end{longtable}

Though evidence of food insecurity was not provided by for the vast
majority of beneficiaries reached, much of the evidence that was
provided were reasonable justifications for targeting beneficiaries.
Some good reasons included armed conflict, community-based beneficiary
selection and the use of the food consumption score.

\begin{longtable}[]{@{}
  >{\centering\arraybackslash}p{(\columnwidth - 4\tabcolsep) * \real{0.24}}
  >{\centering\arraybackslash}p{(\columnwidth - 4\tabcolsep) * \real{0.22}}
  >{\centering\arraybackslash}p{(\columnwidth - 4\tabcolsep) * \real{0.25}}@{}}
\caption{Breakdown of evidence of food insecurity status in 2022/Q1
}\tabularnewline
\toprule
\begin{minipage}[b]{\linewidth}\centering
evidence
\end{minipage} & \begin{minipage}[b]{\linewidth}\centering
beneficiaries
\end{minipage} & \begin{minipage}[b]{\linewidth}\centering
\%\_beneficiaries
\end{minipage} \\
\midrule
\endfirsthead
\toprule
\begin{minipage}[b]{\linewidth}\centering
evidence
\end{minipage} & \begin{minipage}[b]{\linewidth}\centering
beneficiaries
\end{minipage} & \begin{minipage}[b]{\linewidth}\centering
\%\_beneficiaries
\end{minipage} \\
\midrule
\endhead
Armed conflict & 87,590 & 67.78 \\
No evidence & 41,645 & 32.22 \\
\bottomrule
\end{longtable}

The general lack of evidence of evidence of beneficiaries' food
insecurity status makes it difficult to justify to affected communities
and donors that the Food Security Cluster is reaching the most in need.
This highlights the need to promote a shared understanding of the
response through the development of a common prioritisation tool for
food security partners.

\hypertarget{beneficiary-disaggregation}{%
\subsubsection{4.3 Beneficiary
disaggregation}\label{beneficiary-disaggregation}}

Due to the problems in reporting disaggregated beneficiary data, two
tests have been applied to the submitted 5W data. The first involves a
comparison to the proportions of disaggregation groups in the census to
determine if values have been backfilled from the census.

The plots below show the breakdowns between the ``real'' values and
those that have been backfilled from the census. Approximately 59\% of
beneficiaries reported had values that were not backfilled from the
census; this is an improvement from 2021, where only 44\% of
beneficiaries had ``real'' disaggregations.

In the ``real'' values, it can be seen that the proportion of adult
females reached is much higher than adult males -- this is in line with
the Cluster's understanding of several activities that specifically
target women. The percentages of elderly persons actually reached is
also much lower than what has been reported.

\includegraphics{fsc_q1_2022_southeast_word_files/figure-latex/unnamed-chunk-5-1.pdf}

The second test applied is if the disaggregated numbers of beneficiaries
reached have been copied and pasted. To do this, the proportions of each
disaggregation group by partner have been compared to how close they
were to the mean for the entire group. To explain: if partner A reported
that 40\% of beneficiaries in an activity were adult females, this
percentage was then compared to the average percentage of adult females
for all other activities reported by that partner. This measure whether
or not the same proportions were copied and pasted throughout the 5W
form.

It is extremely unlikely that these percentages would be similar across
activities as implementing partners worked in an average of 32.39
locations.

In the plot below, the closer a value is to 0\% on the x-axis, the more
likely it is that it was copied and pasted. It is estimated that 89\% of
beneficiary disaggregation values were copied and pasted.

\includegraphics{fsc_q1_2022_southeast_word_files/figure-latex/unnamed-chunk-6-1.pdf}

\hypertarget{partners}{%
\subsection{5. Partners}\label{partners}}

\hypertarget{reach-by-implementing-partner}{%
\subsubsection{5.1 Reach by implementing
partner}\label{reach-by-implementing-partner}}

There are 36 partners that were involved in direct implementation that
have reported achievements in first quarter of 2022. These implementing
partners corresponded to a total of 15 reporting organisations. The
largest reporting organisation, org\_2690, had 21 implementing partners.
All other reporting organisations had 1 or 2 implementing partners.

The interactive plot below shows the number of beneficiaries and
townships reached by implementing partner.

\includegraphics{fsc_q1_2022_southeast_word_files/figure-latex/plotly-partner-scatter-1.pdf}

In 2021, it was noted that whilst there was much variation in the
numbers of beneficiaries reached by each implementing partner, their
geographic footprints were quite limited. This pattern has continued
into 2022/Q1. Only 6 partners (17\% of the total) have a presence in
more than 5 townships. The distribution of partners remains an
impediment to the implementation of a countrywide response. And the
following steps mentioned in the 2021 report are still very necessary:

\begin{itemize}
\item
  Incentivise partners to expand their footprints
\item
  Identify new partners to reach vulnerable persons in areas recently
  affected by conflict
\item
  Encourage donors to support expansion of Food Security activities in
  areas recently affected by conflict (with sufficient support costs)
\end{itemize}

\hypertarget{monthly-progress-by-partner}{%
\subsubsection{5.2 Monthly progress by
partner}\label{monthly-progress-by-partner}}

\includegraphics{fsc_q1_2022_southeast_word_files/figure-latex/partner-progress-facet-line-1.pdf}

Overall, 13 implementing partners increased the number of beneficiaries
reached over their 2021 totals by more than 50\%; 29 partners who
reported in 2021 also reported in 2022/Q1. 7 new implementing partners
reported in 2022. And 28 partners who reported in 2021 but have not yet
any achievements in 2022.

\texttt{summarise()} has grouped output by `org\_code'. You can override
using the \texttt{.groups} argument.

\begin{longtable}[]{@{}
  >{\centering\arraybackslash}p{(\columnwidth - 10\tabcolsep) * \real{0.15}}
  >{\centering\arraybackslash}p{(\columnwidth - 10\tabcolsep) * \real{0.15}}
  >{\centering\arraybackslash}p{(\columnwidth - 10\tabcolsep) * \real{0.17}}
  >{\centering\arraybackslash}p{(\columnwidth - 10\tabcolsep) * \real{0.15}}
  >{\centering\arraybackslash}p{(\columnwidth - 10\tabcolsep) * \real{0.17}}
  >{\centering\arraybackslash}p{(\columnwidth - 10\tabcolsep) * \real{0.17}}@{}}
\caption{Top implementing partners by beneficiaries reached in
2022/Q1}\tabularnewline
\toprule
\begin{minipage}[b]{\linewidth}\centering
org\_code
\end{minipage} & \begin{minipage}[b]{\linewidth}\centering
ben\_2021
\end{minipage} & \begin{minipage}[b]{\linewidth}\centering
rank\_2021
\end{minipage} & \begin{minipage}[b]{\linewidth}\centering
ben\_2022
\end{minipage} & \begin{minipage}[b]{\linewidth}\centering
rank\_2022
\end{minipage} & \begin{minipage}[b]{\linewidth}\centering
total\_ben
\end{minipage} \\
\midrule
\endfirsthead
\toprule
\begin{minipage}[b]{\linewidth}\centering
org\_code
\end{minipage} & \begin{minipage}[b]{\linewidth}\centering
ben\_2021
\end{minipage} & \begin{minipage}[b]{\linewidth}\centering
rank\_2021
\end{minipage} & \begin{minipage}[b]{\linewidth}\centering
ben\_2022
\end{minipage} & \begin{minipage}[b]{\linewidth}\centering
rank\_2022
\end{minipage} & \begin{minipage}[b]{\linewidth}\centering
total\_ben
\end{minipage} \\
\midrule
\endhead
org\_9566 & NA & NA & 85,590 & 1 & 85,590 \\
org\_7002 & 25,954 & 21 & 18,021 & 2 & 43,975 \\
org\_6130 & 71,467 & 11 & 12,396 & 3 & 83,863 \\
org\_2690 & 103,611 & 8 & 3,590 & 4 & 107,201 \\
org\_7970 & 2,580 & 40 & 203 & 5 & 2,783 \\
org\_1206 & 33,442 & 20 & NA & NA & 33,442 \\
org\_1233 & 5,162 & 30 & NA & NA & 5,162 \\
org\_1538 & 4,208 & 32 & NA & NA & 4,208 \\
org\_2157 & 2,612 & 38 & NA & NA & 2,612 \\
org\_2254 & 41,231 & 16 & NA & NA & 41,231 \\
org\_2441 & 3,697 & 33 & NA & NA & 3,697 \\
org\_2461 & 4,433 & 31 & NA & NA & 4,433 \\
org\_2537 & 6,052 & 29 & NA & NA & 6,052 \\
org\_2807 & 1,001 & 48 & NA & NA & 1,001 \\
org\_2825 & 49,692 & 15 & NA & NA & 49,692 \\
\bottomrule
\end{longtable}

\hypertarget{donors}{%
\subsubsection{5.3 Donors}\label{donors}}

As shown by the table below, the majority of beneficiaries reported in
the first quarter of 2022/Q1 were reported without any corresponding
donor, as in 2021. The data in this column continues of limited utility
in analysis.

\begin{longtable}[]{@{}
  >{\centering\arraybackslash}p{(\columnwidth - 4\tabcolsep) * \real{0.29}}
  >{\centering\arraybackslash}p{(\columnwidth - 4\tabcolsep) * \real{0.22}}
  >{\centering\arraybackslash}p{(\columnwidth - 4\tabcolsep) * \real{0.25}}@{}}
\caption{Top donors by benficiaries reached}\tabularnewline
\toprule
\begin{minipage}[b]{\linewidth}\centering
donor
\end{minipage} & \begin{minipage}[b]{\linewidth}\centering
beneficiaries
\end{minipage} & \begin{minipage}[b]{\linewidth}\centering
\%\_beneficiaries
\end{minipage} \\
\midrule
\endfirsthead
\toprule
\begin{minipage}[b]{\linewidth}\centering
donor
\end{minipage} & \begin{minipage}[b]{\linewidth}\centering
beneficiaries
\end{minipage} & \begin{minipage}[b]{\linewidth}\centering
\%\_beneficiaries
\end{minipage} \\
\midrule
\endhead
FCDO & 43,019 & 33.29 \\
CERF & 29,026 & 22.46 \\
WVI & 20,843 & 16.13 \\
BPRM & 13,457 & 10.41 \\
No donor specified & 10,516 & 8.14 \\
FCDO+BPRM & 7,059 & 5.46 \\
LIFT & 2,898 & 2.24 \\
Norad & 1,900 & 1.47 \\
Other donors & 517 & 0.4 \\
\bottomrule
\end{longtable}

\hypertarget{comparison-with-targets}{%
\subsection{6. Comparison with targets}\label{comparison-with-targets}}

\hypertarget{reached-vs-target-by-township}{%
\subsubsection{6.2 Reached vs target by
township}\label{reached-vs-target-by-township}}

The specifics of each township can be reviewed with the interactive plot
below. Each point is a township, with the size indicating the number of
beneficiaries. The x-axis indicates the target population by township
and the y-axis shows the number of beneficiaries reached in 2022/Q1.

The red line down the middle represents reaching 100\% of the target.
Townships above this line have reached more beneficiaries than their
target and townships below the line have not met their target yet. The
further away a township is from the red line, the further above or below
its target it is. Mouse over each of the townships to see more details.

The 12 townships along the extreme left side of the plot have
beneficiaries but do not have targets (their targets have just been
coded as \(1\) so that they show up on the plot). 230 townships with
targets have not been reached.

\includegraphics{fsc_q1_2022_southeast_word_files/figure-latex/plotly-tsp-comparison-reached-target-1.pdf}

\hypertarget{map-of-beneficiaries-reached-in-2022q1-vs-target}{%
\subsubsection{6.2 Map of beneficiaries reached in 2022/Q1 vs
target}\label{map-of-beneficiaries-reached-in-2022q1-vs-target}}

\includegraphics{fsc_q1_2022_southeast_word_files/figure-latex/maps-ben-target-1.pdf}

With the important exceptions of Yangon and the Southeast, beneficiaries
are concentrated in the peripheral and border regions of the union,
where humanitarian actors have traditionally been present. As mentioned
in previous reports, this is not consistent with the current patterns of
needs and vulnerability.

\hypertarget{interactive-reference-table}{%
\subsubsection{6.3 Interactive reference
table}\label{interactive-reference-table}}

There was an overallocation of resources in these relatively few areas
in 2021 and this has continued in the first quarter of 2022. In the
interactive table below, is a list of townships sorted by the gap
between the targeted population and beneficiaries reached in 2022. Any
of the columns can be sort; the search bars above each column can also
assist in filtering.

\end{document}
